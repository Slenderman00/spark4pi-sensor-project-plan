% !TeX root = Report.tex
\documentclass[12pt]{article}

% Package imports (organized and deduplicated)
\usepackage{biblatex}
\usepackage{changepage}
\usepackage{color}
\usepackage{enumitem}
\usepackage{float}
\usepackage{graphicx}
\usepackage{listings}
\usepackage{sectsty}
\usepackage{xcolor}
\usepackage[breaklinks=true]{hyperref}
\usepackage{xurl}
\usepackage{tikz}
\usepackage{lipsum}
\usepackage[format=plain,
            labelfont=it,
            textfont=]{caption}
\usetikzlibrary{shapes.geometric,positioning,fit,backgrounds}
\usepackage{./timing-diagrams}
\usetikzlibrary{calc}
\setcounter{biburlnumpenalty}{100}
\setcounter{biburlucpenalty}{100}
\setcounter{biburllcpenalty}{100}

\newcommand{\writersnote}[1]{\marginpar{\small{\textcolor{blue}{Writer's note:}} \scriptsize\textit{#1}}}

% \usepackage{background}
% \backgroundsetup{
%   position=current page.north west,
%   angle=0,
%   nodeanchor=north west,
%   vshift=-1cm,
%   hshift=1cm,
%   color=red,
%   opacity=1,
%   scale=1,
%   contents={Preprint}
% }

\definecolor{darkblue}{RGB}{0, 0, 102} 
\hypersetup{
    colorlinks=true,
    pdfborder={0 0 0},
    linkbordercolor=white,
    urlcolor=darkblue,
    linkcolor=darkblue,
    citecolor=darkblue,
    filecolor=darkblue
}

% Make bibliography ragged right instead of justified
\AtBeginDocument{
  \renewcommand{\bibsetup}{\raggedright}
}
% Document configuration
\restylefloat{table}
\graphicspath{{./images/}}
\addbibresource{Library.bib}
\subsectionfont{\fontsize{12}{14}\selectfont}

% Author information
\author{
    Group Number: 107\\
    Joar Heimonen, Christian Vu, Naly Keli \\
    \texttt{contact@joar.me}\\ 
    \texttt{chvu002@student.kristiania.no}\\
    \texttt{nake002@student.kristiania.no}
}

% Title configuration
\title{
    \textbf{Spark4pi-sensors}\\[0.5em]
    \large Project description
}
\date{\today}

\newcommand{\license}{
    \vspace{1em}
    \noindent\small{© 2024 Joar Heimonen\\
    This work is licensed under a \href{https://creativecommons.org/licenses/by-sa/4.0/}{Creative Commons Attribution-Sharealike 4.0 International License}.}
}
\begin{document}
\maketitle
\pagebreak

\tableofcontents

\pagebreak


\section{Introduction}
This document will cover the software and hardware implementation of the Spark4pi-sensors shield, with an emphasis
on software. The Spark4pi-sensors is a shield developed by Lightside Instruments for the deployment and testing of arbitrary I2C sensors.
The system is designed to be scalable and reliable through YANG datamodels and the NETCONF protocol.

\section{Technical background}
\subsection{Power Over Ethernet (PoE)}
Power over Ethernet (PoE) is a technology that allows network cables to carry electrical power. 
There are two standards for PoE, IEEE 802.3af and IEEE 802.3at \cite{IEEEStandardsAssociation}. 
The ethernet4pi-zero PoE hat uses IEEE 802.3af which 
is the older standard and allows for a maximum power draw of 15.4w.

\subsection{Raspberry Pi Zero}
The Raspberry Pi Zero is a single board computer developed by the Raspberry Pi Foundation \cite{foundationTeachLearnMake2025}. 
The Raspberry Pi Zero is the smallest and cheapest Raspberry Pi.

\subsection{Ethernet4pi-zero PoE}
The ethernet4pi-zero PoE is a PoE network shield in the late stages of development.
The shield is designed to be used with the Raspberry Pi Zero and the Raspberry Pi Zero W.
Ethernet4pi-zero PoE is built around a Lightside-Instruments powermod-tps54260 which delivers a maximum of five volts at three amps.
This is equal to a maximum power draw of 15 watts which is the maximum power draw of the IEEE 802.3af standard.

\subsection{Powermod-tps54260}
The powermod-tps54260 is a DC to DC voltage regulator that can deliver a maximum of five volts at three amps.
It works with an input voltage between 10 and 60 volts. The powermod-tps54260 is designed to be modular and
can be placed on any project that requires a stable five-volt power supply. It is based on the Texas Instruments TPS54260 \cite{TPS54260DataSheet}.

\subsection{RJ45}
An RJ45 connector is a type of connector commonly used on Ethernet cables.

\subsection{Sensirion RJ45 sensor interface}
Sensirion uses the RJ45 connector to implement their proprietary sensor interface.
This interface is designed to be used with their reference sensor implementations. Through 
an adapter called the SEK-SensorBridge \cite{agSEKSensorBridgeConnectingBridgeTwo}.

\subsection{YANG}
YANG is a data modeling language, similar to XML. It is used to model and configure the state of network controlled devices \cite{bjorklundYANG11Data2016}
YANG implements a hierarchical model of what the device can do. Unlike SNMP this model is fetched from the device.

\subsection{NETCONF}
NETCONF is a protocol that is used to configure and monitor network devices. It uses the YANG data modeling language to describe the configuration and state of network devices.
NETCONF can be implemented over any protocol but is usually implemented over SSH \cite{ennsNetworkConfigurationProtocol2011}.

\subsection{I2C}
I2C is a two wire bus protocol used to communicate between devices. Each device has an address on the bus and can be read from and Written to.
Only one device can communicate on the bus at a time. The I2C bus is a master slave bus, the master controls the bus and the slaves respond to the master.
I2C stands for Inter-Integrated Circuit and was developed by Philips in 1982, now known as NXP \cite{I2CbusSpecificationUser2021}.

\subsection{Field Programmable Gate Array (FPGA)}
A field programmable gate array is a hardware language defined integrated circuit \cite{FieldprogrammableGateArray2025}. 
Much like a microcontroller an FPGA is programmed. FPGA's are programmed using a hardware description language like VHDL or Verilog.
The difference between an FPGA and a microcontroller is that when an FPGA is programmed we are defining 
the states of the field gates and the connections between them. This makes an FPGA much more flexible than a microcontroller.

\subsection{XPath}
XPath is a query language designed to query and format XML. \cite{XPathMDN2024}

\subsection{Grafana}
Grafana is an open source tool for visualization of arbitrary data. Grafana allows users to create
dashboards consisting of widgets. Using "Grafana integrations" Grafana is able to fetch data from almost any 
system. \cite{GrafanaOpenComposable}

\section{Lightside Instruments}
"Lightside Instruments AS develops instruments with model based network management for use in networking, 
network interconnect testing and telemetry. Designed with YANG (RFC7950) model and NETCONF (RFC6241) protocol. 
Based on IETF standards and drafts. Implemented with software tools available in Debian. programmable logic and open hardware." \cite{LightsideInstrumentsYANG}

\subsection{Use Case}
This product will be the first sensor platform developed with NETCONF in mind. 
It is developed to be a reference implementation of our/IETF's sensor model, and will
be referenced by future designs. It will also be deployed at Bitraf to monitor air quality.

\section{Design}
\subsection{Hardware}
\subsubsection{Philosophy}
We believe that simplicity is the key to robust systems, 
the Spark4pi-sensors platform is designed following this simple philosophy based on the 
Unix philosophy which favors modularity over monolithic design \cite{BasicsUnixPhilosophy}.
Each of Lightside Instruments AS products must therefor meet the following criteria:
\begin{itemize}
    \item Do one thing and do it well.
    \item Interopperate with other Lightside Instruments AS products.
    \item Present a NETCONF/YANG interface.
\end{itemize}
Consisting only of a power delivery system, an FPGA, twelve pull-up resistors and six RJ45 connectors. 
We have created a design that meets these criteria.
\\
\\
Through this simple philosophy Lightside Instruments AS has created an ecosystem of interoperating products that can be
leveraged in numerous ways.

\subsubsection{Ice4pi FPGA}
The ice4pi is a Field Programmable Gate Array (FPGA) shield designed by Lightside Instruments AS.
This FPGA will provide the six I2C buses required by the Spark4pi-sensors platform. 
The ice4pi will communicate with the Raspberry Pi through an SPI interface.

\subsection{Software}
\subsubsection{YANG}
A YANG model will be developed for sensors, 
We will also consider the viability of using existing models that implement support for sensors.
The model must support the following:
\begin{itemize}
  \item An infinite ammount of sensors.
  \item Arbitraty model defined sensors.
  \item Predefined common sensors.
\end{itemize}

\subsubsection{NETCONF}
A NETCONF module will be developed using C and Python. This module will support an arbitrary ammount of sensors,
making it hardware agnostic.

\subsubsection{Grafana}
A NETCONF plugin will be developed for Grafana. 
This plugin will prompt NETCONF endpoints at a regular interval and store it in a timeseries database.
This will make the data searchable trough XPath.

\section{Summary}
The Spark4pi-sensors is a reference implmenetation of a NETCONF sensor platform. 
The platform is based on raspberry pi zero and a ice4pi FPGA shield. It has six I2C buses, 
allowing it to read data from up to six identical sensors. It features a YANG datamodel that is sensor agnostic
and allows for an infinite ammount of sensors. A grafana plugin will be developed to visualize data from NETCONF devices.

\pagebreak
\addcontentsline{toc}{section}{References}
\printbibliography
\license
\end{document}
