% !TeX root = Report.tex
\documentclass[12pt]{article}

% Package imports (organized and deduplicated)
\usepackage{biblatex}
\usepackage{changepage}
\usepackage{color}
\usepackage{enumitem}
\usepackage{float}
\usepackage{graphicx}
\usepackage{listings}
\usepackage{sectsty}
\usepackage{xcolor}
\usepackage[breaklinks=true]{hyperref}
\usepackage{xurl}
\usepackage{tikz}
\usetikzlibrary{shapes.geometric,positioning,fit,backgrounds}
\usepackage{./timing-diagrams}
\usetikzlibrary{calc}
\setcounter{biburlnumpenalty}{100}
\setcounter{biburlucpenalty}{100}
\setcounter{biburllcpenalty}{100}

% \usepackage{background}
% \backgroundsetup{
%   position=current page.north west,
%   angle=0,
%   nodeanchor=north west,
%   vshift=-1cm,
%   hshift=1cm,
%   color=red,
%   opacity=1,
%   scale=1,
%   contents={Preprint}
% }

\definecolor{darkblue}{RGB}{0, 0, 102} 
\hypersetup{
    colorlinks=true,
    pdfborder={0 0 0},
    linkbordercolor=white,
    urlcolor=darkblue,
    linkcolor=darkblue,
    citecolor=darkblue,
    filecolor=darkblue
}

% Make bibliography ragged right instead of justified
\AtBeginDocument{
  \renewcommand{\bibsetup}{\raggedright}
}
% Document configuration
\restylefloat{table}
\graphicspath{{./images/}}
\addbibresource{Library.bib}
\subsectionfont{\fontsize{12}{14}\selectfont}

% Author information
\author{
    Joar Heimonen\\
    \texttt{contact@joar.me}
}

% Title configuration
\title{
    \textbf{Spark4pi-sensors}\\[0.5em]
    \large Project Plan
}
\date{\today}

\newcommand{\license}{
    \vspace{1em}
    \noindent\small{© 2024 Joar Heimonen\\
    This work is licensed under a \href{https://creativecommons.org/licenses/by-sa/4.0/}{Creative Commons Attribution-Sharealike 4.0 International License}.}
    \vspace{1em}
}

\begin{document}
\maketitle

\begin{abstract}
    \noindent Spark4pi-sensors is a hardware platform for analog, i2c and SPI sensors. 
    This document describes the preliminary hardware schematic and the development plan for the Spark4pi-sensor platform. 
    Spark4pi-sensors implements six rj45 sensor interfaces based on Sensirons reference implementation. 
    Three different power busses are implemented. Groups of two sensors share one of the four available
    power buses, each group also shares an i2c bus.
\end{abstract}

\pagebreak

\tableofcontents

\pagebreak


\section{Introduction}
When designing a sensor platform, universality is of the highest importance. This document will describe the preliminary implementation of Spark4pi-sensors and the development plan.
Spark4pi-sensors is a hardware platform for analog, i2c and SPI sensors. The platform is designed to support a wide range of sensors powered by a wide range of voltages.
The platform is designed to be rack mounted taking up one unit in a 19" rack. This platform is powered by a 12V power supply our trough PoE using the ethernet4pi-zero PoE hat.
The platform will be controlled and configured through a NETCONF/YANG interface. 


\section{Technical background}
\subsection{Power Over Ethernet (PoE)}
Power over Ethernet (PoE) is a technology that allows network cables to carry electrical power. 
There are two standards for PoE, IEEE 802.3af and IEEE 802.3at. The ethernet4pi-zero PoE hat uses IEEE 802.3 af which 
is the older standard and allows for a maximum power draw of 15.4w.

\subsection{Ethernet4pi-zero PoE}
The ethernet4pi-zero PoE is a PoE network shield in the late stages of development.
The shield is designed to be used with the Raspberry Pi Zero and the Raspberry Pi Zero W.
Ethernet4pi-zero PoE is built around a Lightside-Instruments powermod-54260 which delivers a maximum of five volts at three amps.
This is equal to a maximum power draw of 15 watts which is the maximum power draw of the IEEE 802.3af standard.

\subsection{Powermod-54260}
The powermod-54260 is a DC to DC voltage regulator that can deliver a maximum of five volts at three amps.
It works with an input voltage between 10 and 60 volts. The powermod-54260 is designed to be modular and
can be placed on any project that requires a stable five-volt power supply.

\subsection{RJ45}
A RJ45 connector is a type of connector commonly used on Ethernet cables.

\subsection{Sensirion RJ45 sensor interface}
Sensirion uses the RJ45 connector to implement their proprietary sensor interface.
This interface is designed to be used with their reference sensor implementations. Through 
an adapter called the SEK-SensorBridge.

\pagebreak
\addcontentsline{toc}{section}{References}
\printbibliography
\license
\end{document}